\documentclass[12pt,a4paper]{article}

% -------------------------------------------------------------
% PACKAGES
% -------------------------------------------------------------
\usepackage{geometry}
\usepackage{graphicx}
\usepackage{titlesec}
\usepackage{xcolor}
\usepackage{fancyhdr}
\usepackage{setspace}
\usepackage{parskip}
\usepackage{enumitem}
\usepackage{tcolorbox}
\usepackage{tabularx}

\geometry{margin=1in}
\setlength{\headheight}{15pt}
% -------------------------------------------------------------
% COLORS — Corporate Slate Theme
% -------------------------------------------------------------
\definecolor{slate}{HTML}{2F3A45}
\definecolor{accent}{HTML}{4A6572}
\definecolor{soft}{HTML}{ECEFF1}

% -------------------------------------------------------------
% SECTION STYLE
% -------------------------------------------------------------
\titleformat{\section}
  {\Large\bfseries\color{slate}}
  {\thesection}{1em}{}

\titleformat{\subsection}
  {\large\bfseries\color{accent}}
  {\thesubsection}{1em}{}

% -------------------------------------------------------------
% HEADER / FOOTER
% -------------------------------------------------------------
\pagestyle{fancy}
\fancyhf{}
\lhead{\textcolor{slate}{Tech Solutions (SARL) — Network Infrastructure}}
\rhead{\textcolor{slate}{\thepage}}

% -------------------------------------------------------------
% TITLE PAGE
% -------------------------------------------------------------
\begin{document}

\begin{titlepage}
    \centering

    \vspace*{2cm}
    \includegraphics[width=0.55\linewidth]{../../img/CORE/sarl.png}

    \vfill

    {\Huge\bfseries Network Infrastructure Architecture Report \\[0.5cm]}
    {\Large Tech Solutions (SARL)}\\[1.2cm]

    {\large Version 1.0 — \today}\\[0.5cm]

    \vfill
\end{titlepage}

\newpage

% -------------------------------------------------------------
% EXECUTIVE SUMMARY
% -------------------------------------------------------------
\section*{Executive Summary}

This document provides a complete professional overview of the network infrastructure designed for Tech Solutions (SARL). 
The architecture simulates a modern enterprise network using GNS3, featuring routed backbone segments, VLAN segmentation, firewall policies, and Linux-based services.

The objective is to model a scalable, secure, and realistic environment for engineering,
training, and testing purposes.  
It provides system engineers and network architects with a clear understanding of the network's logical and physical design, implementation steps, and operational considerations.

\newpage

% -------------------------------------------------------------
% TABLE OF CONTENTS
% -------------------------------------------------------------
\tableofcontents
\newpage

% -------------------------------------------------------------
% 1. INTRODUCTION
% -------------------------------------------------------------
\section{Introduction}

The Tech Solutions (SARL) network architecture project is a fully simulated enterprise environment developed within GNS3. It includes multi-router infrastructure, departmental VLANs, dynamic routing, and server resources.

The purpose of this report is to present:

\begin{itemize}
  \item network design principles,
  \item device roles and interconnections,
  \item used technologies and protocols,
  \item deployment and implementation steps,
  \item configuration and testing methodology.
\end{itemize}

\newpage

% -------------------------------------------------------------
% 2. NETWORK OVERVIEW
% -------------------------------------------------------------
\subsection{Departmental Allocation and Logical Topology}

After the integration of the Internet service, the ISP allocated the address space \textbf{172.24.0.0/14} to the enterprise.  
The logical topology and IP allocation for each department are summarized below:

\begin{center}
\begin{tabularx}{\textwidth}{|>{\raggedright\arraybackslash}p{3cm}|
                            >{\raggedright\arraybackslash}p{2cm}|
                            >{\raggedright\arraybackslash}p{2.5cm}|
                            >{\raggedright\arraybackslash}p{2cm}|
                            >{\raggedright\arraybackslash}X|}
\hline
\textbf{Department} & \textbf{Local Router} & \textbf{Service / VM} & \textbf{PC Clients} & \textbf{Role in the Enterprise} \\
\hline
Web / Marketing & RZ-1 & Web Server & 8905 & Manages the website and customer-facing interactions \\
\hline
Supervision / IT & RZ-2 & Monitoring Server & 1465 & Monitors server and network performance \\
\hline
Base de données / Gestion & RZ-3 & Database Server & 489 & Centralizes and secures client and internal data \\
\hline
Partage / Collaboration & RZ-4 & NFS Server & 165 & Enables internal file and document sharing \\
\hline
\end{tabularx}
\end{center}

\subsection{Subnetting (VLSM) Diagram}

\begin{center}
\includegraphics[width=\linewidth]{../../img/CORE/sarl-vlsm.png}
\end{center}


\newpage

% -------------------------------------------------------------
% 3. TECHNOLOGIES USED
% -------------------------------------------------------------
\section{Technologies Used}

\begin{itemize}
  \item \textbf{GNS3 3.0.5} — Network simulation environment.
  \item \textbf{Cisco Routers (3725/3745)} — Core routing infrastructure.
  \item \textbf{Ubuntu Server 20.04} — Server and network service hosting.
  \item \textbf{OSPF} — Dynamic routing protocol.
  \item \textbf{VLANs} — Logical segmentation and broadcast containment.
  \item \textbf{Nmap \& Wireshark} — Analysis and testing tools.
  \item \textbf{ZeroTier} — Potential remote access overlay network.
  \item \textbf{SSH / Telnet} — Device management protocols.
\end{itemize}

\newpage

% -------------------------------------------------------------
% 4. ARCHITECTURE DESIGN
% -------------------------------------------------------------
\section{Architecture Design}

\begin{center}
\includegraphics[width=\linewidth]{../../img/CORE/tech-solutions-sarl-network-infrastructure.png}
\end{center}

\subsection{Core Layer}

The backbone contains the primary routing equipment enabling communication between all segments.

\subsection{Distribution Layer}

VLAN segmentation and policy enforcement occur at this layer.

\subsection{Access Layer}

End-user devices and servers connect at this level.

\newpage

% -------------------------------------------------------------
% 5. IMPLEMENTATION STEPS
% -------------------------------------------------------------
\section{Implementation Steps}

\begin{enumerate}
  \item Prepare GNS3 environment and import appliance templates.
  \item Deploy routers, switches, and servers.
  \item Configure VLANs and interfaces.
  \item Configure OSPF across backbone routers.
  \item Implement ACLs and security policies.
  \item Verify connectivity and routing.
  \item Test bandwidth, latency, and redundancy.
\end{enumerate}

\newpage

% -------------------------------------------------------------
% 6. SECURITY MODEL
% -------------------------------------------------------------
\section{Security Model}

Security policies were applied on multiple layers:

\begin{itemize}
  \item ACLs restricting cross-department traffic.
  \item VLAN isolation.
  \item SSH-enabled secure device management.
  \item Logging and monitoring using Linux servers.
\end{itemize}

\newpage

% -------------------------------------------------------------
% 7. CONCLUSION
% -------------------------------------------------------------
\section{Conclusion}

This document summarizes the complete network architecture for Tech Solutions (SARL).  
The simulated environment provides a realistic, scalable, and secure platform for training, experimentation, and enterprise design validation.

\vfill

\begin{center}
\textbf{End of Report}
\end{center}

\end{document}
